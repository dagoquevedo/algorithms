\documentclass[letterpaper,11pt]{article}
\usepackage[utf8]{inputenc}
\usepackage[spanish]{babel}
\usepackage{mathtools}
\usepackage{latexsym}
\usepackage{xcolor}
\usepackage{multicol}
\usepackage{tikz}
\usetikzlibrary{arrows}
\usetikzlibrary {positioning}
\definecolor {processblue}{cmyk}{0.96,0,0,0}


\begin{document}

\title{Modelos de Computación\\\large Máquinas de Turing\\\large Actividad 3}
\author{Dagoberto Quevedo}
\maketitle

\begin{abstract}
Una máquina de Turing puede simular cualquier algoritmo a través de una sola estructura de datos: sucesión de símbolos escrita en una cinta infinita, con operaciones de escritura y eliminación de símbolos.

Formalmente una máquina de Turing se describe como sigue $M = (Q,s,r,\Sigma,\delta)$, donde $Q$ es un conjunto finito de estados, $s$ un estado de aceptación, $r$ un estado de rechazo, $\Sigma$ es un conjunto finito de símbolos, llamado el alfabeto de $M$, que contiene dos símbolos especiales $\rhd$ y $\sqcup$, $D$ conjunto de direcciones $\leftarrow$ (izquierda), $\rightarrow$ (derecha), $\textendash$ (sin mover), la función,
\begin{equation}
\delta(q,\sigma)=(p,\rho,d),
\end{equation}
es una función de transición de $M$, donde $q\in Q\cup\{s,r\}$ es el estado actual y $\sigma\in\Sigma$ es el símbolo actual en el puntero, $p\in Q\cup\{s,r\}$ es el nuevo estado, $\rho$ el nuevo símbolo que es escrito en reemplazo de $\sigma$ en la posición actual y $d\in D$ es la dirección hacia donde se mueve el puntero \cite{Schaeffer2020}.

En esta actividad se proporcionan las funciones de transición $\delta$ para dos máquinas de Turing de reconocimiento de lenguaje.

\end{abstract}


\section{Máquina de Turing A}
Proporcione una función de transición $\delta$ para una máquina de Turing que determina si su entrada $x$ contiene una $a$ después de una $b$, tal que $q_1\in Q$ es el estado inicial, \textcolor{blue}{$q_f$} un estado de aceptación, \textcolor{red}{$q_e$} un estado de rechazo y $\Sigma = \{\rhd,a,b,c,\sqcup\}$ el alfabeto.

\break

\subsection{Función de transición $\delta$}
\begin{multicols}{2}
\begin{enumerate}
\item $(q_1,\rhd) = (q_1,\rhd,\rightarrow)$
\item $(q_1,a) = (q_1,a,\rightarrow)$
\item $(q_1,c) = (q_1,c,\rightarrow)$
\item $(q_1,b) = (q_2,b,\rightarrow)$
\textcolor{blue}{\item $(q_2,a) = (q_f,a,\textendash)$}
\item $(q_2,b) = (q_2,b,\rightarrow)$
\item $(q_2,c) = (q_2,c,\rightarrow)$
\textcolor{red}{\item $(q_1,\sqcup) = (q_e,\sqcup,\textendash)$}
\textcolor{red}{\item $(q_2,\sqcup) = (q_e,\sqcup,\textendash)$}
\end{enumerate}
\end{multicols}

\subsubsection{Representación gráfica}

\begin{figure}[h!]
\centering
\begin {tikzpicture}[-latex ,auto ,node distance =4 cm and 5cm ,on grid ,
semithick ,
state/.style ={ circle ,
draw, minimum width =1 cm}]
\node[state] (q1) {$q_1$};
\node[state] (q2) [right=of q1] {$q_2$};
\node[state] (qf) [below=of q2] {\textcolor{blue}{$q_f$}};
\node[state] (qe) [below=of q1] {\textcolor{red}{$q_e$}};

\path (q1) edge [loop left] node[left] {$a$} (q1);
\path (q1) edge [loop above] node[above] {$c$} (q1);
\path (q1) edge [loop right] node[right] {$\rhd$} (q1);
\path (q1) edge [bend left =25] node[above =0.15 cm] {$b$} (q2);
\path (q2) edge [loop right] node[right] {$b$} (q2);
\path (q2) edge [loop above] node[above] {$c$} (q2);
\path (q2) edge [bend left =0] node[right = 0.15 cm] {$a$} (qf);
\path (q1) edge [bend right =0] node[left = 0.15 cm] {$\sqcup$} (qe);
\path (q2) edge [bend right =0] node[left = 0.15 cm] {$\sqcup$} (qe);
\end{tikzpicture}
\end{figure}

\subsection{Casos de prueba}

\begin{table}[h!]
\centering
\begin{tabular}{c|c|c|c|c}
\hline
$x$ & $q$ & $p$ & $\rho$ & $d$ \\ \hline
\textcolor{blue}{\underline{$\rhd$}}acbbac$\sqcup$  & $q_1$ & $q_1$ & $\rhd$ & $\rightarrow$  \\
$\rhd$\textcolor{blue}{\underline{a}}cbbac$\sqcup$  & $q_1$ & $q_1$ & a & $\rightarrow$  \\
$\rhd$a\textcolor{blue}{\underline{c}}bbac$\sqcup$  & $q_1$ & $q_1$ & c & $\rightarrow$  \\
$\rhd$ac\textcolor{blue}{\underline{b}}bac$\sqcup$  & $q_1$ & $q_2$ & b & $\rightarrow$  \\
$\rhd$acb\textcolor{blue}{\underline{b}}ac$\sqcup$  & $q_2$ & $q_2$ & b & $\rightarrow$  \\
$\rhd$acbb\textcolor{blue}{\underline{a}}c$\sqcup$  & $q_2$ & \textcolor{blue}{$q_f$} & a & $\textendash$  \\\hline
\end{tabular}
\caption{Caso con estado final aceptable \textcolor{blue}{$q_f$}, donde $x=acbbac$}
\end{table}

\begin{table}[h!]
\centering
\begin{tabular}{c|c|c|c|c}
\hline
$x$ & $q$ & $p$ & $\rho$ & $d$ \\ \hline
\textcolor{blue}{\underline{$\rhd$}}acbbcc$\sqcup$  & $q_1$ & $q_1$ & $\rhd$ & $\rightarrow$  \\
$\rhd$\textcolor{blue}{\underline{a}}cbbcc$\sqcup$  & $q_1$ & $q_1$ & a & $\rightarrow$  \\
$\rhd$a\textcolor{blue}{\underline{c}}bbcc$\sqcup$  & $q_1$ & $q_1$ & c & $\rightarrow$  \\
$\rhd$ac\textcolor{blue}{\underline{b}}bcc$\sqcup$  & $q_1$ & $q_1$ & b & $\rightarrow$  \\
$\rhd$acb\textcolor{blue}{\underline{b}}cc$\sqcup$  & $q_1$ & $q_2$ & b & $\rightarrow$  \\
$\rhd$acbb\textcolor{blue}{\underline{c}}c$\sqcup$  & $q_2$ & $q_2$ & c & $\rightarrow$  \\
$\rhd$acbbc\textcolor{blue}{\underline{c}}$\sqcup$  & $q_2$ & $q_2$ & c & $\rightarrow$  \\
$\rhd$acbbcc\textcolor{blue}{\underline{$\sqcup$}}  & $q_2$ & \textcolor{red}{$q_e$} & $\sqcup$ & $\textendash$  \\\hline
\end{tabular}
\caption{Caso con estado final rechazado \textcolor{red}{$q_e$}, donde $x=acbbcc$}
\end{table}

\break

\section{Máquina de Turing B}

Proporcione una función de transición $\delta$ para una máquina de Turing que identifica si una cadena proveniente de un alfabeto $\Sigma$ es o no un palíndromo, tal que $q_1\in Q$ es el estado inicial, \textcolor{blue}{$q_f$} un estado de aceptación, \textcolor{red}{$q_e$} un estado de rechazo y $\Sigma = \{\rhd,a,b,\sqcup\}$ el alfabeto.

\subsection{Función de transición $\delta$}
\begin{multicols}{2}
\begin{enumerate}
\item $(q_1,\rhd) = (q_1,\rhd,\rightarrow)$
\item $(q_1,a) = (q_2,\rhd,\rightarrow)$
\item $(q_2,a) = (q_2,a,\rightarrow)$
\item $(q_2,b) = (q_2,b,\rightarrow)$
\item $(q_2,\sqcup) = (q_3,\sqcup,\leftarrow)$
\item $(q_3,a) = (q_4,\sqcup,\leftarrow)$
\textcolor{red}{\item $(q_3,b) = (q_e,b,\textendash)$}
\item $(q_4,a) = (q_4,a,\leftarrow)$
\item $(q_4,b) = (q_4,b,\leftarrow)$
\item $(q_4,\rhd) = (q_1,\rhd,\rightarrow)$
\item $(q_1,b) = (q_5,\rhd,\rightarrow)$
\item $(q_5,a) = (q_5,a,\rightarrow)$
\item $(q_5,b) = (q_5,b,\rightarrow)$
\item $(q_5,\sqcup) = (q_6,\sqcup,\leftarrow)$
\item $(q_6,b) = (q_7,\sqcup,\leftarrow)$
\textcolor{red}{\item $(q_6,a) = (q_e,a,\textendash)$}
\item $(q_7,a) = (q_7,a,\leftarrow)$
\item $(q_7,b) = (q_7,b,\leftarrow)$
\item $(q_7,\rhd) = (q_1,\rhd,\rightarrow)$
\textcolor{blue}{\item $(q_1,\sqcup) = (q_f,\sqcup,\textendash)$}

\end{enumerate}
\end{multicols}

\break
\subsubsection{Representación gráfica}


\begin{figure}[h]
\centering
\begin {tikzpicture}[-latex ,auto ,node distance =3 cm and 3cm ,on grid ,
semithick ,
state/.style ={ circle ,
draw, minimum width =1 cm}]
\node[state] (q1) {$q_1$};
\node[state] (q2) [right=of q1] {$q_2$};
\node[state] (q5) [left=of q1] {$q_5$};
\node[state] (q6) [below=of q5] {$q_6$};
\node[state] (qf) [below=of q1] {\textcolor{blue}{$q_f$}};
\node[state] (q3) [below=of q2] {$q_3$};

\node[state] (q7) [below=of q6] {$q_7$};
\node[state] (qe) [below=of qf] {\textcolor{red}{$q_e$}};
\node[state] (q4) [below=of q3] {$q_4$};

\path (q1) edge [bend left =0] node[above =0.15 cm] {$a$} (q2);
\path (q1) edge [bend right =0] node[above =0.15 cm] {$b$} (q5);

\path (q2) edge [bend left =0] node[right =0 cm] {$\sqcup$} (q3);
\path (q5) edge [bend left =0] node[left =0 cm] {$\sqcup$} (q6);
\path (q1) edge [bend left =0] node[left =0 cm] {$\sqcup$} (qf);

\path (q6) edge [bend left =0] node[left =0 cm] {$a$} (qe);
\path (q3) edge [bend left =0] node[right =0 cm] {$b$} (qe);

\path (q6) edge [bend left =0] node[left =0 cm] {$b$} (q7);
\path (q3) edge [bend left =0] node[right =0 cm] {$a$} (q4);


\path (q1) edge [loop above] node[above] {$\rhd$} (q1);

\path (q5) edge [loop above] node[above] {$a$} (q5);
\path (q5) edge [loop left] node[left] {$b$} (q5);

\path (q2) edge [loop above] node[above] {$a$} (q2);
\path (q2) edge [loop right] node[right] {$b$} (q2);

\path (q7) edge [loop left] node[left] {$a$} (q7);
\path (q7) edge [loop below] node[below] {$b$} (q7);

\path (q4) edge [loop right] node[right] {$a$} (q4);
\path (q4) edge [loop below] node[below] {$b$} (q4);

\path (q7) edge [bend left =0] node[left =0 cm] {$\rhd$} (q1);
\path (q4) edge [bend left =0] node[right =0 cm] {$\rhd$} (q1);

\end{tikzpicture}
\end{figure}

\subsection{Casos de prueba}

\begin{table}[h!]
\centering
\begin{tabular}{c|c|c|c|c}
\hline
$x$ & $q$ & $p$ & $\rho$ & $d$ \\ \hline
\textcolor{blue}{\underline{$\rhd$}}abbb$\sqcup$  & $q_1$ & $q_1$ & $\rhd$ & $\rightarrow$  \\
$\rhd$\textcolor{blue}{\underline{a}}bbb$\sqcup$  & $q_1$ & $q_2$ & $\rhd$ & $\rightarrow$  \\
$\rhd\rhd$\textcolor{blue}{\underline{b}}bb$\sqcup$  & $q_2$ & $q_2$ & b & $\rightarrow$  \\
$\rhd\rhd$b\textcolor{blue}{\underline{b}}b$\sqcup$  & $q_2$ & $q_2$ & b & $\rightarrow$  \\
$\rhd\rhd$bb\textcolor{blue}{\underline{b}}$\sqcup$  & $q_2$ & $q_2$ & a & $\rightarrow$  \\
$\rhd\rhd$bbb\textcolor{blue}{\underline{$\sqcup$}}  & $q_2$ & $q_3$ & $\sqcup$ & $\leftarrow$  \\
$\rhd\rhd$bb\textcolor{blue}{\underline{b}}$\sqcup$  & $q_3$ & \textcolor{red}{$q_e$} & b & $\textendash$  \\\hline
\end{tabular}
\caption{Caso con estado final rechazado \textcolor{red}{$q_e$}, donde $x=abbb$}
\end{table}

\break
\begin{table}[h!]
\centering
\begin{tabular}{c|c|c|c|c}
\hline
$x$ & $q$ & $p$ & $\rho$ & $d$ \\ \hline
\textcolor{blue}{\underline{$\rhd$}}abba$\sqcup$  & $q_1$ & $q_1$ & $\rhd$ & $\rightarrow$  \\
$\rhd$\textcolor{blue}{\underline{a}}bba$\sqcup$  & $q_1$ & $q_2$ & $\rhd$ & $\rightarrow$  \\
$\rhd\rhd$\textcolor{blue}{\underline{b}}ba$\sqcup$  & $q_2$ & $q_2$ & b & $\rightarrow$  \\
$\rhd\rhd$b\textcolor{blue}{\underline{b}}a$\sqcup$  & $q_2$ & $q_2$ & b & $\rightarrow$  \\
$\rhd\rhd$bb\textcolor{blue}{\underline{a}}$\sqcup$  & $q_2$ & $q_2$ & a & $\rightarrow$  \\
$\rhd\rhd$bba\textcolor{blue}{\underline{$\sqcup$}}  & $q_2$ & $q_3$ & $\sqcup$ & $\leftarrow$  \\
$\rhd\rhd$bb\textcolor{blue}{\underline{a}}$\sqcup$  & $q_3$ & $q_4$ & $\sqcup$ & $\leftarrow$  \\
$\rhd\rhd$b\textcolor{blue}{\underline{b}}$\sqcup\sqcup$  & $q_4$ & $q_4$ & b & $\leftarrow$  \\
$\rhd\rhd$\textcolor{blue}{\underline{b}}b$\sqcup\sqcup$  & $q_4$ & $q_4$ & b & $\leftarrow$  \\
$\rhd$\textcolor{blue}{\underline{$\rhd$}}bb$\sqcup\sqcup$  & $q_4$ & $q_1$ & $\rhd$ & $\leftarrow$  \\
$\rhd\rhd$\textcolor{blue}{\underline{b}}b$\sqcup\sqcup$  & $q_1$ & $q_5$ & $\rhd$ & $\rightarrow$  \\
$\rhd$$\rhd$$\rhd$\textcolor{blue}{\underline{b}}$\sqcup\sqcup$  & $q_5$ & $q_5$ & b & $\rightarrow$  \\
$\rhd$$\rhd$$\rhd$b\textcolor{blue}{\underline{$\sqcup$}}$\sqcup$  & $q_5$ & $q_6$ & $\sqcup$ & $\leftarrow$  \\
$\rhd$$\rhd$$\rhd$\textcolor{blue}{\underline{b}}$\sqcup\sqcup$  & $q_6$ & $q_7$ & $\sqcup$ & $\leftarrow$  \\
$\rhd$$\rhd$\textcolor{blue}{\underline{$\rhd$}}$\sqcup$$\sqcup$$\sqcup$  & $q_7$ & $q_1$ & $\rhd$ & $\leftarrow$  \\
$\rhd$$\rhd$$\rhd$\textcolor{blue}{\underline{$\sqcup$}}$\sqcup$$\sqcup$  & $q_1$ & \textcolor{blue}{$q_f$} & $\sqcup$ & $\textendash$  \\\hline
\end{tabular}
\caption{Caso con estado final aceptable \textcolor{blue}{$q_f$}, donde $x=abba$}
\end{table}

\begin{thebibliography}{0}
  \bibitem{Schaeffer2020} Elisa Schaeffer, \textit{Modelos computacionales}, Complejidad computacional de problemas y el análisis y diseño de algoritmos, notas de curso, 2020.
\end{thebibliography}

\end{document}