%-------------------------------------------


\documentclass[letterpaper,11pt]{article}

\begin{document}

Grafos: dado un valor $m$ que determina el numero de elementos en $V$, generar un grafo no dirigido $G=(V,E)$, donde las aristas $(i,j)\in E, i,j\in V$ se asignan desde una probabilidad uniforme de conexión entre sus elementos definida por $p_{ij}$. Una vez generado calcular lo siguiente: a) diámetro, usando la ecuación (1.84), densidad, formula (1.83) y c) determinar el elemento más y menos central del grafo a partir de la centralidad de grado, donde el grado de un elemento $i$ es dado por $\deg(i) = \sum_{j\in V} A_{ij}$, siendo $A$ es la matriz de adyacencias.

\end{document}

%-------------------------------------------
